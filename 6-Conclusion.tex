
\chapter[Conclusion \& Future Direction]{Conclusion \&\\ Future Direction}
\label{ch:conc}

%In this thesis, I proposed the idea of changing body schema mapping with an artificial representation in a systematic manner in order to achieve alternative ways of interaction through mediated digital representations.

In this thesis, I proposed the idea of changing body schema mapping embody new functions through a mediated representations. This thesis outlined the concept of Embodied-Driven Design to create a range of embodied oriented applications through a series of use-cases and meta-models. In this chapter, a summary recapping the used methodologies of this research, outcome contributions, and current limitations. Finally, this thesis is concluded with an outlook to the future direction. %

\section{Summary}

The topic of body schema modification and transfer have been studied before in philosophy to understanding human cognition of his/her body. Various cognitive driven theories have been proposed in this regard, and few of them has been put into evaluation and testing. Most of these studies were focused on testing specific phenomena of body perception. This thesis carries these theoretical studies and places them into a practical model that is derived from human body modalities. This work argues that human sense of his body schema can be altered and restructured by first overcoming the rigid coupling of his various modalities, and then use a meta-modeling approach to define the new model of the body schema. This approach aims to bridge between human body and digital representations to create transparent interaction through them.
%which are considered the building blocks for meta-level presence modeling. The first concept is Loosely-Coupled Modalities (LCM) that defines a designing approach to decoupling and recoupling body modalities into smaller, highly manageable modalities blocks that can be viewed as input or output channels to the human body. By decoupling these channels of modalities, its possible to overcome the physical constraints on the human body, and thus can achieve a higher level of abstraction and mapping. The second concept is an Embodied-Driven Design (EDD) modeling approach that builds on LCM concept of modalities blocks. 

This work approaches body schema altering by creating a topology of human body modalities, that act as input and output channels of the body. These modalities are represented as model blocks that offer channels of information to communicate with the human body and can be altered using a meta-model. EDD uses this meta-model to define the relationships between the various modalities used by the user or the target representation (physical or virtual) to define mapping and transfer functions between these blocks. EDD uses three types of modalities blocks to model the body schema: Perceptual, Transfer, and Representation blocks. The meta-model interfaces with both the user via various types of tracking and feedback blocks, and with the used representation using communication blocks. Three types of designing categories summarize the general use cases of the proposed meta-modeling: Representation alteration, Operation modulation, and Perception model expansion. Each category is concerned with a different type of body schema and perception mapping. Detailed design considerations of EDD has been thoroughly discussed in \Chapter{ch:concept}.

\Chapter{ch:impl} provided the implementation details for EDD framework. A meta-modeling editing environment has been developed and integrated into a commonly used tool, Unity3D, to facilitate the process of designing body schema models. The meta-modeling editor provides three blocks categories that can be directly used in the meta-model, or new meta-blocks can be implemented depending on the design requirements of the target application. These blocks provided in the editor have a wide support for various types of tracking and feedback technologies for the user side. To facilitate the design and prototyping process, a physical embodiment toolkit was developed. This toolkit provides the essential input/output modalities to create remote presence experiences and can be used directly within the meta-modeling environment. An optimization method for visual feedback ``Foveated Streaming'' integrated into the toolkit, which significantly enhances the overall performance, especially when multiple representations with visual feedback were required (such as Layered Presence case).

The proposed framework, meta-modeling approach, and toolkit were used to resolve five different use-cases addressing body limitations and resolving physical disability. \Chapter{ch:eval} showed the possibility to remodel body schema in a manner that allowed the users (depending on the use-case) to adapt to new body mapping that: extends their spatial interaction, alters their body representation, expands their perceptual feedback, or reconfigure their body postural topology. The first use-case in \Section{sec:eval-txSys} addressed is a remotely operated surveillance vehicle which is targeted to be used in hazardous situations human cannot access. EDD provided a schematic matching between the operating user and the vehicle system, and assisted to extend spatial connectivity to the representation used. A direct body transfer meta-model approach was used to map operator's body with the corresponding representation, facilitating the embodied control of the robot without requiring adaptation time. The second use-case in \Section{sec:eval-enforced} explored the possibility to use different mean to represent the body to overcome physical constraints and arm reach. Virtual arms were used to substitute physical representation, and to expand the spatial interactions of the arms by projecting them on the representation side. Representation alteration meta-modeling approach was used to facilitate this design. Third use-case discussed in \Section{sec:eval-hug} showed the effectiveness of using EDD to enable a physically impaired person to use the disabled function of her body. For this project, a topology reconfiguration meta-model was used to redirect a functioning modality (eye-gaze) to the disabled modality (neck), allowing the lady to use her eyes to control the head motion of the representation. Due to the correlated of eye and head, the lady was capable to adapt to the new mapping, and actively participate at a remote location. Fourth use-case discussed in \Section{sec:eval-layeredpresence} discussed the possibility to expand visual awareness from one location to many, leveraging sense of presence to ubiquitous level. Body schema and sensory topology are mapped to several representations simultaneously, which the corresponding feedback modalities are layered and provided to the user achieving parallel awareness of multiple locations. Modality expansion meta-model outlined the relationship between body and representations used. The last use-case explored the possibility to reconfigure our body postural schema, and discussed in \Section{sec:eval-metalimbs}. In this use-case, we demonstrated the plasticity human body has to adapt to new limb mapping, and to use it as an extension of the body instead of being a tool. MetaLimbs system was developed and used to expand the number of the arms. The meta-model redirects the feet motion to the arms trajectory, thus the user can maintain a sense of embodiment and agency towards the newly added arms. 

Through these use-cases, we demonstrated the effectiveness of Embodied-Driven Design approach to alter the association between human and digital representations (or tools) from a meta-level perspective, to instead of using the tools, rather embodying their actions and feedback for effective and minimal adaptation. 



%Finally, EDD framework has been used in social context to support remote presence applications. In \Chapter{ch:social}, two different use-cases EDD had contributed in the design process. 

\section{Contribution}

The design of body extension tools and technologies are necessary to be transparent and seamless with our actions in order to overcome a limitation or augment body functions. This thesis tackles this topic by providing an embodied oriented design approach with a modeling environment to allow body schema meta-modeling. Here I revise the main contributions and outcomes which this thesis provided:

\begin{itemize}
\item This thesis explored the theories of body and embodiment and investigated in the topic of the role of technology on our body and perception. A systematic model for body schema was established that translates the theories of body plasticity into practical tools. Embodied-Driven Design presented an abstraction medium for body modalities rewiring and design.

\item Meta-modeling framework was designed and developed that realizes the concept of Embodied-Driven Design and has been integrated into a commonly used platform to enable a wide range of embodied oriented applications. This meta-modeling framework shifts the design process of the conventional body augmentation systems to become a top level approach, that the design of the interaction is inherited from body capabilities and functions, and is possible to alter depending on the subjective use. Also, this framework can be effectively used prior the actual design of the system, so the designer could prototype and simulate the required mapping using virtual representation before deploying/designing a physical one.

\item Four categories for body schema meta-modeling were proposed to facilitate meta-modeling design process: direct body transfer, representation alteration, topology reconfiguration, and modality expansion. These meta-models summarize the general scenarios which the EDD can modulate body schema to overcome a limitation or disability. These categories were evaluated using five scenarios.

%\item Embodied Driven Design (EDD) framework was used to realize two practical social and industrial projects. The first project is HUG project as a social contribution, enabling physically disabled person to actively join a remote event. And Telexistence based Surveillance System as an industrial contribution, used for hazardous environment access.

\item A physical embodiment toolkit was designed and developed that provides essential functions for remote presence applications. Toolkit functions were integrated into the EDD framework which can be accessed and mapped with user's body. This toolkit allows rapid prototyping for the embodiment designers to use and test of their ideas.

\end{itemize}

\section{Limitations}

Currently, the developed framework does not provide any constraints on the design process and the flow of information between the used modalities within the meta-model. This unconstrained design process is highly flexible during the modeling process, however, depending on the design used, the result interaction could generate unnatural and inconsistent feedback. For example, in HUG Project, when switching the eye gaze mapping axis in a way horizontal motion is mapped to robot's head tilt, and vertical motion to the pan axis, it would result in a highly unnatural feedback to the visual-motor system regardless of the correlated mapping. Although such design-error scenarios can be validated by experiencing the model and iterate until the inconsistency is resolved. Another approach is to deploy an embodiment grammar into the design process which can validate whether the model would result in such unnatural interaction. Thus reducing the space of possible meta-models down to a valid set of mapping operations that maintains the embodied interaction of the meta-model.

For the body operation modulation use-cases, such as discussed in MetaLimbs \ref{sec:eval-metalimbs}, the system lacks proprioceptive feedback regarding the posture of the artificial arms. Since the system as was shown uses an open-loop structure with no muscle feedback, the user relies on visual modality to understand the displaced posture of the arms when they are forced to a different posture externally (such as handshaking). We will further investigate using EMS feedback \cite{lopes2015proprioceptive} in such operation modulation scenarios to produce proprioceptive cues to the body, creating a closed-loop system.


\section[Outlook]{Outlook: Towards Radical Bodies}
%\section{Final Remarks}

When writing this thesis, I established the topic of Embodied-Driven Design as a step of defining a systematic approach to leverage the limitations of human body through mediated tools based on the philosophy of human body and psychological studies conducted before in this area. Radical Bodies, the introduction topic of this thesis has presented the effect of the exponential progression of technology in reshaping our cognitive and physical abilities in tasks we were limited at. Some thinkers believes that this reliance on technology will not only be as a tool, but rather we would become the tools and lose our essence:

\begin{quote}
    
``Every day millions of people decide to grant their smartphone a bit more control over their lives or try a new and more effective antidepressant drug. In pursuit of health, happiness and power, humans will gradually change first one of their features and then another, and another, until they will no longer be human.''
― Yuval Harari, Homo Deus: A Brief History of Tomorrow p. 56
\end{quote}

Despite the what can be seen as a pessimistic prediction of the progress of human augmentation, there are essential boundaries that can be used as a framework to shape this progress, such as using human-driven approaches to maintain the ownership and agency. Embodied Driven Design builds on these boundaries to define a framework for human augmentation tools design and altering the body schema. These presented tools are aimed to maintain the essence of the embodiment as the human is the core and the purpose of the design.

Radical Bodies in general aims to further investigate the role of technology and its impact on our life and body. How to maintain transparency between our bodies and the computational/robotic means to enhance our well-being and efficiency. Radical Bodies does not mean to change the structure of our bodies or mutating them, but rather to use the tools as complementary to our bodies in a non-invasive manner. 

The following topics outline future development and investigations for Radical Bodies:

\subsubsection*{Embedded \& Embodied Communication}

One of the major problems in human bodies augmentation is realizing new communication channels for sensing and feedback that can be customized and integrated with the tools.

For the side of the coin, is the sensing part. Two paradigms of embodied tracking can be used: the explicit and internal state. The first approach (explicit) to use our voluntary controlled muscles\footnote{healthy body has over 650 voluntary skeletal muscles.} as an input mechanism in the design process, or to use their spatial end position (posture of the arm for example), which have been addressed in the design chapter. Other approaches investigated this problem by measuring the internal state of the body and mapping it to some actions, such as BCI (Brain Computer Interface) research. BCI monitors brain cortices activities for a specific set of features and matched with pre-trained limited catalog of actions (such as move left, right, trigger on, off, ...etc), which can be used as an activation mechanism to external tools and devices. However, our embodied actions are rather continuous and do not fall into a discrete set of commands. The use of hybrid methods (mind-body driven approaches) however, could lead to a wider expansion of the toolset of continuous control. I intend to further explore this area for embodied interaction and design. 

The other side of the coin is the feedback part. As has been discussed in Body Operation Modulation: MetaLimbs \Section{sec:eval-metalimbs}, when we modulate the body operation using an external device, we would need to maintain the sensory feedback to the body. In that case, it is the proprioceptive feedback. The mechanism and type of stimulation used to the body play an important role in identifying whether the medium belongs to our body, or as being an external object. It enhances the subjective sensation of ownership toward the medium. This thesis addressed the feedback mapping for vision, auditory, and tactile modalities, however further exploration in spatial feedback is required. EMS is one of the approaches that can support this spatial transfer and manipulation to the proprioceptive state of the human body.

\subsubsection*{Beyond Tools}

The cycle of human and robotics has been carried along many iterations which were developing a better understanding of the role of technology in our life. The concept of Radical Bodies stretches beyond the use of technology as tools, but rather as a way to reshape our bodies and augment us in a new spectrum of tasks and activities. Recent topic, Superhuman Sports \cite{kunze2017superhuman}, has addressed the role of technology in changing our bodily capabilities in sports or augmenting our sensory feedback. The further the technology is embodied into us, the less we are aware of it, and more engaged we become in the activity at hand. As has been demonstrated in this thesis, Embodied-Driven Design supports such an argument by designing activities and tools driven by our bodily capabilities, and amplifying them using digitized representations.


