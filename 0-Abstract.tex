\eabstract{
%telexistence and telepresence are considered as an enabling technologies
%how to bridge between the rigid, physical, biological bodies with the currently evolving technologies
%present the idea of rewiring human body, perception, and its modalities into digital representation.
%talk about the idea of I/O channels in human body, and how to use it to direct the perception flow
%challenge that our bodies and cognition can be plasticity and able to adapt to new forms

\footnotesize
Our bodies are inherently constrained by their physical structure and are limited in the perceptual spectrum which we use to perceive and act through. We use tools and devices to alter or overcome certain physical characteristics and capacities of our bodies temporarily or permanently, enhancing our body functions beyond the biological limits. Augmented Human, a recent research paradigm, was established addressing body augmentation using technological driven approaches, accelerating the process of machine-mediated interaction. However, design approach of such interactions has been problem driven, focusing on solving certain bodily limitations. There have been no systematic approaches for a body driven design that allows the re-usability, or re-configuration of human body interaction through the digital means. How to design intuitive, natural connections to digital means so they can be integrated and embodied into our bodies and actions? 

This thesis addresses such a question by proposing an embodied driven methodology for body interaction with digital means. \textit{Embodied-Driven Design (EDD)} is an approach to configure body representation and mapping with the augmentative digital means. EDD explores the plasticity of the human body to adapt to a new schema, and proposes a systematic design methodology to achieve such new body schema. Using the proposed approach, an abstract model of the human body is used to define the flow between body's modalities (body's sensory organs and joints) with a target representation (referring to a digital mean) creating a new body schema. Each of these modalities acts as information channels of the human body as well as of the representation(s). The information flow can be altered, mapped, or combined between the modalities within the meta-model, reflecting the results as embodied actions through the used representation. 

Embodied-Driven Design framework was designed and implemented to realize the proposed approach. The framework introduces body schema building blocks that act as transfer functions of body's modalities information (sensory and spatial). Meta-model defines the relationships and the flow between these building blocks with the target representation(s). To facilitate body schema design process, four meta-modeling categories were driven from the framework: \textit{Direct Body Transfer}, \textit{Representation Alteration}, \textit{Topology Reconfiguration}, and \textit{Modality Expansion} meta-models. The first category focuses on directly mapping body modalities with the representation maintaining their information flow without alteration. The second category addresses body physical limitations and constraints by modifying its physical attributes through the representation. The third category reconfigures body topology and mapping with the representation to overcome a physical disability or limitation. The last category focuses on expanding modalities' information flow into multiple representations, achieving non-linear body mapping. 

The use of EDD meta-modeling approaches in body augmentation has been highlighted through addressing five different use-cases to overcome body physical limitations or disabilities. By changing the body mapping through the meta-model, we showed the possibility to: stretch the spatial awareness of the body (discussed in Telexistence based Surveillance System), substitute body with a virtual one to expand its physical limits (discussed in Enforced Mutual Presence), substitute a disabled body modality with another one (discussed in HUG Project), reconfigure body limbs to an gain extra two arms (discussed in MetaLimbs), and to expand modality awareness to multiple locations achieving ubiquitous presence (discussed in Layered Presence). These five projects showed the plasticity of human body and cognition to adapt to new modality configurations.

%we show the possibility to configure the body representation to combine a virtual representation of user's arms along an existing Telexistence system to expand his arm's reach in a remote location (Enforced Mutual Presence). In the second category, body schema is modulated to incorporate extra two arms by mapping legs modalities into the robotic arms (MetaLimbs). For the last category, body visual and auditory modalities are mapped into multiple Telexistence systems to expand user's perception into multiple locations with simultaneous awareness (Layered Presence). This categorization facilitates the process of designing embodied mapping through the intended body schema. 
%Using this meta-modeling modeling approach, it is possible to design body mapping to overcome the physical constraints of the body. We show a social impact of using EDD to enable a physically disabled person in operating a substitute avatar system (HUG Project). Also, an industrial use-case is discussed to map operator's body into a remote vehicle for hazardous situations (Telexistence based Surveillance System). 

Embodied-Driven Design can be used in human augmentation, supportive and assistive body extensions, and beyond spatial interaction. Using EDD, it is possible to further explore body plasticity in a systematic manner, and use the existing knowledge of embodiment in designer oriented approach. We hope Embodied-Driven Design can help at radically changing the design approach of human augmentation, and to assist researchers and designers in prototyping and evaluating their ideas.



}